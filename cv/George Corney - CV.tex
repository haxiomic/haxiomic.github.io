%%%%%%%%%%%%%%%%%%%%%%%%%%%%%%%%%%%%%%%%%
% Medium Length Professional CV
% LaTeX Template
% Version 2.0 (8/5/13)
%
% This template has been downloaded from:
% http://www.LaTeXTemplates.com
%
% Original author:
% Trey Hunner (http://www.treyhunner.com/)
%
% Important note:
% This template requires the resume.cls file to be in the same directory as the
% .tex file. The resume.cls file provides the resume style used for structuring the
% document.
%
%%%%%%%%%%%%%%%%%%%%%%%%%%%%%%%%%%%%%%%%%

%----------------------------------------------------------------------------------------
%	PACKAGES AND OTHER DOCUMENT CONFIGURATIONS
%----------------------------------------------------------------------------------------

\documentclass{resume} % Use the custom resume.cls style

\usepackage[left=0.75in,top=0.6in,right=0.75in,bottom=0.6in]{geometry} % Document margins
\usepackage[colorlinks=true, linkcolor=blue, urlcolor=blue]{hyperref}



\name{George Corney} % Your name
\address{28 Ganna Park Road \\ Plymouth \\ PL3 4NN} % Your address
\address{07542 045079 \\ \href{mailto:george@superfluid.co}{george@superfluid.co}} % Your phone number and email

\begin{document}


\begin{rSection}{Introduction}
% Creative Technologist
\item Creative technologist, comfortable working across multiple domains from native development, through rendering engineering, realtime networking to design and interactive art.
\item Founding engineer at \href{https://lumalabs.ai/dream-machine}{lumalabs.ai}. Previous clients include Microsoft, Nike, LG, LUSH, Atlantic Productions and the Met Office. My work both personal and contracted has reached millions of users and featured on sites like \href{http://thenextweb.com/creativity/2015/05/15/webgl-fluid-experiment-is-a-browser-based-lsd-trip/}{The Next Web}, \href{http://www.gizmodo.co.uk/2014/11/just-try-and-stop-playing-with-this-fluid-simulator/}{Gizmodo} and \href{http://www.fastcodesign.com/3038725/this-wonderful-web-toy-turns-your-browser-into-magic-liquid}{FastCoDesign}.

\item I focus on GPU rendering and web technologies, however, I'm never shy to grapple with a new domain to meet the demands of a project. 

\item I'm an advocate of open source and you can find my contributions under the handle \href{http://github.com/haxiomic}{`haxiomic'}.

\end{rSection}

%----------------------------------------------------------------------------------------
%	WORK EXPERIENCE SECTION
%----------------------------------------------------------------------------------------

\begin{rSection}{Selected Experience}

\begin{rSubsection}{Refik Anadol – dataland.art}{September 2024 - Present}{WebGL, GLSL, three.js, Next.js, React, TypeScript}{Contracting}
\item Developed an \href{https://x.com/refikanadol/status/1869051351130382519}{interactive particle artwork} as the centerpiece for their ``Large Nature Model", an AI system that generates visions of nature. The artwork visually represents the AI's hyperdimensional parameters.
\item As the AI generates new images, the particle system dynamically morphs into 3D representations of nature, creating a seamless connection between the model's output and the visual experience.
\end{rSubsection}



\begin{rSubsection}{lumalabs.ai}{Nov 2022 - September 2024}{WebGL, GLSL, C++, WebAssembly, three.js, Next.js, React, TypeScript}{FTE}
\item Founding engineer at \href{https://lumalabs.ai}{lumalabs.ai}. Developed the web editor for their neural radiance field rendering tool, featured in a \href{https://www.youtube.com/watch?v=YX5AoaWrowY}{Corridor Crew video}.
\item Worked on rendering engineering for the web-based real-time Gaussian splat renderer. Including developing \href{https://www.npmjs.com/package/@lumaai/luma-web}{three.js integration}.
\item Developed interactive artworks \& contributions to product release websites.
\end{rSubsection}

\begin{rSubsection}{Nike Rise – Interactive Flagship Store}{May 2021 - July 2021}{WebGL, GLSL, JavaScript}{Contracting}
\item Working with Nike \& \href{https://field.io/work/nike-rise-intelligent-retail-system}{field.io} to develop an \href{https://field.io/work/nike-rise-intelligent-retail-system}{interactive store experience} driven by live data and customer interaction. Alongside developing data-driven live artworks throughout the store I worked on the \href{https://hypebeast.com/2021/8/nike-rise-seoul-retail-concept-store-info}{reactive entrance archway} – an doorway of LED displays. As customers walk through the visuals react to their motion. I implemented the piece including developing the computer vision algorithms to extract motion from video.
\end{rSubsection}

\begin{rSubsection}{LUSH – Interactive Installation \& Other Projects}{March 2019 - August 2020}{React, WebGL, GLSL, C++, emscripten, iOS, Android, TypeScript, 3D}{Contracting}
\item LUSH commissioned an \href{https://twitter.com/Haxiomic/status/1146820050445393921?s=20}{interactive fluid art piece} to be projected on a wall within a new flagship store opening in Shinjuku Japan. In addition, the piece was to be embedded as an experience within in their native mobile app \href{https://apps.apple.com/gb/app/lush-labs/id1439333565}{LUSH Labs} (iOS and android)
\item The mobile app build mandated small binary sizes and close-to-the-metal performance, the in-store interactive required integrating with native C++ vision SDK and the project schedule necessitated a rapid turnaround time.
\item I used GLSL to implement optical-flow tracking and the interactive fluid simulation and haxe to build cross-platform native OpenGL views that could be embedded easily within native and web apps with a low-footprint. This approach represents an alternative to heavy game engines like Unity and slower-iteration languages like C++ or Rust. It's an approach I'm trying to pioneer and develop \href{https://github.com/haxiomic/haxe-c-bridge}{open source projects} to make it easier.
\item Other projects involved compiling FFmpeg to WebAssembly so WebGL$\rightarrow$MP4 video encoding could be performed in-browser, React for building UIs to host WebGL experiences and Babylon.js for PBR rendering.
\end{rSubsection}

\begin{rSubsection}{VALIS - Genomics Startup}{December 2017 - March 2019}{React, WebGL, GLSL, Text Rendering, TypeScript}{Contracting}
\item Developing a WebGL-based genomics data viewer, capable of navigating through gigabytes of data in realtime. Traditional genomics data is often heavy to load and slow navigate, this project involved developing a pipeline to enable streaming and rendering genomics data so it can be explored interactively within a web browser.
\item To achieve best performance, a \href{https://github.com/VALIS-software/Engine}{WebGL UI, animation} and \href{https://github.com/VALIS-software/GPUText}{text rendering system} was developed from the ground up.
\item The VALIS viewer is now in-use at the \href{https://www.encodeproject.org/}{ENCODE project} to preview genomics data in the browser.
\end{rSubsection}

\begin{rSubsection}{Microsoft}{June 2016 - August 2017}{WebGL, GLSL, Physically Based Rendering, C++}{Contracting}
\item Working on Microsoft's internal cross-platform physically-based rendering engine used in products including Office, Windows 10 shell and Paint3D.
\item I developed the PBR pipeline for the WebGL version of the engine and much of my work on the WebGL engine is now available in \href{https://www.babylonjs.com/demos/pbrglossy/}{BablyonJS}.
\item During this time also I worked on optimizing rendering and launch performance in MS Paint's successor, Paint3D.
\end{rSubsection}

\begin{rSubsection}{Alchemy VR}{Feb 2016 - June 2016}{VR, WebVR, C++, three.js, WebGL, GLSL, node.js}{Contracting}
\item \href{http://www.alchemyvr.com/}{Alchemy VR} are transitioning to distributing their VR films directly to users (rather then through exhibitions), I was contracted to develop the cross-platform Netflix-style VR video player and VR storefront to enable this.
\item To provide a cross-platform base (where bundling a webview wasn't an option) I implemented the WebGL API over GLES using the V8 JavaScript engine (C++), this enabled us to reap the benefits of developing for the web (live feedback, easy distribution, easier to hire developers etc), whilst allowing access to native code and APIs where necessary.
\item WebVR was in the early experimental phase but much of the specification had been outlined. I used the early specification as a roadmap for implementing the native VR app, so that transitioning to in-browser WebVR would be a relatively painless process
\item I was also responsible for developing the media delivery systems, which involved writing a content delivery server in node.js and developing an in-house DRM system.
\end{rSubsection}

\begin{rSubsection}{Atlantic Productions}{Dec 2015 - Feb 2016}{three.js, WebGL, GLSL, HTML, SCSS}{Contracting}
\item Contracted to develop interactive features for a companion website to \href{http://www.atlanticproductions.tv/}{Atlantic Productions} recent series ``David Attenbrough's Great Barrier Reef''. Features included a \href{http://haxiomic.github.io/demos/webgl-earth-v4/}{WebGL globe displaying animated wind flow, atmospheric temperature and ocean current data}
\end{rSubsection}

\begin{rSubsection}{DinahMoe}{Nov 2015}{WebGL, GLSL, three.js, Front-end Web}{Contracting}
\item Contracted by \href{http://dinahmoe.com/}{DinahMoe} to develop portions of an interactive WebGL film, Canada Goose's ``Out There''. The project included developing a system to render WebGL output to video.
\end{rSubsection}

\begin{rSubsection}{LG \& Responsive Ads}{Sept 2015}{WebGL, GLSL, Front-end Web}{Contracting}
\item As part of an \href{http://www.lg.com/uk/oled-tv}{LG OLED TV} ad campaign, I was contracted to produce an interactive WebGL fluid and particle simulation in collaboration with \href{http://www.responsiveads.com/}{responsiveads}.
\item The ad was required to run smoothly in all modern browsers, including mobiles and underpowered devices. The computationally intensive nature of the simulation required heavy optimization and an adaptive quality system.
\item Traditionally this sort of simulation requires extensions to WebGL (such as floating point textures), to enable maximum compatibility, techniques were developed to pack simulation data into 4 byte textures which were available in every WebGL instance (whereas extensions are not).
\item A early preview version of the ad is \href{http://haxiomic.github.io/ig783wghhnod/new-brush/}{available}.
\end{rSubsection}

\begin{rSubsection}{Met Office}{June 2015}{WebGL, GLSL}{Consulting}
\item The \href{http://www.informaticslab.co.uk/}{Met Office Informatics Lab} were developing a browser-based 3D visualization of live weather in the UK. I was brought in to consult on techniques for high-performance volumetric rendering and approaches to resolve performance issues their initial WebGL ray marching implementation.
\end{rSubsection}

\begin{rSubsection}{Alchemy VR}{June 2015 - Sept 2015}{Java, C++, Android, Mobile}{Contracting}
\item \href{http://www.alchemyvr.com/}{Alchemy VR} is a recently formed branch of \href{http://www.atlanticproductions.tv/}{Atlantic Productions}, their VR premier ``David Attenborough's First Life'' was to be shown on 80 Gear VRs in the Natural History Museum from the 12th of June (2015).
\item Alchemy asked me to solve a number of critical problems two weeks before their premier deadline. I was tasked with developing a custom VR video player and bypassing the built-in Oculus Home without resorting to rooting the devices.
\item The custom VR video player was developed using C++, Java Native Interface and Oculus's Mobile SDK and the bypass was developed with Java and the Android API.
\end{rSubsection}

%------------------------------------------------

\begin{rSubsection}{fffunction}{Apr 2015}{JavaScript, backbone.js, HTML, SCSS}{Contracting}
\item \href{http://fffunction.co/}{fffunciton} is a digial design agency in the Southwest with clients that include \href{http://www.roland.co.uk}{Roland UK} and the Bristol museums group (\href{http://bristolmuseums.org.uk}{BMGA}).
\item I was brought in to contribute to a browser-based book reader and preview app (commissioned by Oxford University Press). The app was developed with backbone.js, node.js and Grunt. 
\item My role involved developing a page layout engine and viewer thumbnail alongside bug fixes.
\end{rSubsection}

%------------------------------------------------

\begin{rSubsection}{Hive}{Nov 2013 - Jul 2014}{Objective-C, OS X Reverse Engineering,  UI \& UX, JavaScript}{Startup}
\item Hive was a team collaboration app I worked on with a small group during university. It was an experiment in developing the ideal collaboration tool. The philosophy was that team cohesion could be improved by reducing boundaries between computers; the goal was to be able to push content (including running programs) from one device to a teammate's immediately and intuitively (in a similar manner to moving windows between multiple displays).
\item The project won the Cisco Open Collaboration Challenge and was accelerated for 3 months at \href{http://www.dotforge.com/}{dotforge}.
\item My role in the project was development of `State-sharing' (which involved reverse engineering OS X's state saving feature), UI \& UX design, and the development of the native OS X app.
\item I left the project after the release of OS X Yosemite which contained features (Continuity and Handoff) which competed with our core technology `State-sharing'.
\end{rSubsection}

%%%% END Experience

\end{rSection}



%----------------------------------------------------------------------------------------
%	PROJECTS
%----------------------------------------------------------------------------------------

\begin{rSection}{Selected Open Source Projects}

\begin{rSubsection}{WebGL Fluid}{September 2014}{GLSL, Haxe, Lime, JavaScript, C++, WebGL}{}
\item \href{https://github.com/haxiomic/GPU-Fluid-Experiments}{This project} is a GPU fluid and particle simulation written in Haxe and GLSL, targeting HTML5 for browsers and C++ for desktop and iOS. The simulation solves the Navier-Stokes equation for incompressible flow over a grid with the Jacobi method and uses the velocity field to advect over 1 million particles.
\item The motivation for this project was to explore using WebGL for high performance physics simulations and to investigate the performance factors involved.
\item It's been played with approximately \textbf{2 million times} by \textbf{1.6 million users}, achieving a total of \textbf{8 million} pageviews.
\item It has reached the front page of \href{https://www.reddit.com/r/InternetIsBeautiful/comments/2gkunq/fluid_and_particles_in_webgl/}{Reddit} (\href{https://www.reddit.com/r/InternetIsBeautiful/comments/35s6hg/in_browser_physics_simulator_xpost_pc_master_race/}{twice}) and featured in articles on
\begin{itemize}
	\item \href{http://www.fastcodesign.com/3038725/this-wonderful-web-toy-turns-your-browser-into-magic-liquid}{FastCoDesign}
	\item \href{http://thenextweb.com/creativity/2015/05/15/webgl-fluid-experiment-is-a-browser-based-lsd-trip/}{The Next Web}
	\item \href{http://www.gizmodo.co.uk/2014/11/just-try-and-stop-playing-with-this-fluid-simulator/}{Gizmodo}
	\item \href{http://www.engadget.com/2015/05/15/GPU-physics-trippy-simulation/}{engadget}
\end{itemize}
\end{rSubsection}

\begin{rSubsection}{GLSL Parser in Haxe}{Mar 2015 - Present}{GLSL, Haxe, JavaScript, C, Context Free Grammars, LALR}{Work in Progress}
\item The aim of this project is to provide a cross-platform GLSL parser (and parser generator) that supports the GLSL reference language grammar.
\item The motivation for the development was to enable compile-time transformations (such as minification or transpilation) of GLSL source, as well as tighter integration with the host codebase when working with the Haxe compiler.
\item The project can be accessed and tested on github \href{https://github.com/haxiomic/haxe-glsl-parser}{github.com/haxiomic/haxe-glsl-parser}.
\end{rSubsection}
	
\end{rSection}

%----------------------------------------------------------------------------------------
%	EDUCATION SECTION
%----------------------------------------------------------------------------------------

\begin{rSection}{Education}

{\bf University of Sheffield} 
%\hfill {\em June 2004}
\\
B.Sc in Physics with Astrophysics $\diamond{}$ \textit{First Class}
\end{rSection}


%----------------------------------------------------------------------------------------
%	TECHNICAL STRENGTHS SECTION
%----------------------------------------------------------------------------------------

\begin{rSection}{Technical Strengths}

\begin{tabular}{ @{} >{\bfseries}l @{\hspace{6ex}} l }
Programming Languages & JavaScript, GLSL, C++, C, Objective-C, Haxe \\
Technologies \& APIs & HMTL5, WebGL, Git \\
Platforms & Android, iOS, OS X, Linux  \\
Skills & UI \& UX design, Reverse Engineering \\
\end{tabular}

\end{rSection}

%----------------------------------------------------------------------------------------
%	Awards
%----------------------------------------------------------------------------------------

\begin{rSection}{Awards}

\begin{rSubsectionSimple}{BAFTA - Best Digital Creativity - David Attenborough’s Great Barrier Reef Dive (Atlantic Productions)}{February 2017}{Android, Oculus, VR}{}
\end{rSubsectionSimple}

\begin{rSubsectionSimple}{Sudo Challenge VR/AR hackathon 1st place}{November 2017}{WebGL, pixi.js, medical technology}{}
\end{rSubsectionSimple}

\end{rSection}


\end{document}
